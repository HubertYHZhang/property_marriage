\documentclass[11pt]{article}

\usepackage[margin=1.1in]{geometry}
\usepackage{setspace}
\usepackage{amsmath}
% \usepackage{xeCJK}
\usepackage{graphicx}
\usepackage{libertine}
\usepackage{natbib}

\title{Housing Price, Property Rights, and Intra-household Bargaining Power}
\author{Yuhao Zhang}
\date{May 2023}

\onehalfspacing

\begin{document}
\maketitle
\begin{abstract}
    
\end{abstract}

\section{Introduction}
A number of studies explore the impact of changes in housing, a major component of marriage-and-family-related matters in China, on marriage and household behavior. For one thing, traditional notions emphasize housing as the material root of a family. For another thing, the housing market in China has experienced a dramatic boom in the past decades, which has led to a significant increase in housing prices, rendering housing purchase a major challenge for the new couples to conquer. 

In this study, we try to make some subtle contribution to these studies by examining how the rising housing price and marital property right reform co-

\section{Literature Review}
\citet{SUN2020102492} examined how the surging housing price reinforces assortative mating and may enlarge inequality. \citet{WANG2014192} examined how property ownership of females 



\bibliographystyle{apalike}
\bibliography{reference.bib}

\end{document}